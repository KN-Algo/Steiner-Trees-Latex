\documentclass[a4paper,12pt]{article}

% opcje
\input{latex-settings.tex}

\begin{document}

% Nagłówek z logo
\begin{minipage}{0.48\textwidth}
% Miejsce na logo koła naukowego
% Wstaw logo koła naukowego w lewym górnym rogu
\includegraphics[width=0.4\textwidth]{KN_Algo.png}
\end{minipage}
\hfill
\begin{minipage}{0.48\textwidth}
\raggedleft
% Miejsce na logo Politechniki
% Wstaw logo uczelni w prawym górnym rogu
\includegraphics[width=0.4\textwidth]{PWR-pion_Obszar-roboczy-1-1.png}
\end{minipage}

\vspace{2cm} % Przerwa między nagłówkiem a resztą treści

% Tytuł dokumentu
% Miejsce na tytuł dokumentu
\begin{center}
\Large \textbf{Tytuł Dokumentu} \\ % <-- Podaj tytuł dokumentu
\vspace{0.5cm}
\normalsize Opis lub podtytuł dokumentu % <-- Podaj podtytuł lub opis dokumentu
\end{center}


% Treść główna
% Miejsce na główną treść dokumentu

\section{Problem Statement}

% Sekcja brudnopisow - na luzne notateczki, do wyrzucenia przy finalnym produkcie
\input{BrudnopisBartek}

% Bibliografia
\pagestyle{bibliographyStyle}
\bibliographystyle{abbrv}
\bibliography{bibliography}

\end{document}